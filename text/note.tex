\documentclass{article}
\title{Galaxy completeness in two massive fiber allocation setups}
\begin{document}
\maketitle
\section{Introduction}

\section{Fibers and Galaxies}

\subsection{Fibers}

The fibers are placed in an hexagonal tile following a pattern where
the inter-fiber distance, the pitch $P$, is the same for all
fibers. The pattern is shown in Figure 1. 

Each fiber tile is completely described by the patrol radius $r_{\rm
  p}$ and the exclusion radius $r_{\rm e}$.


We try to different setups the 
\begin{itemize}
\item {\rm Despec}. Patrol radius $r_{\rm p}=P$. Exclusion radius
  $r_{\rm e}=1/10\times P$
\item {\rm BB}. Patrol radius $r_{\rm p}=P/\sqrt{3}$. Exclusion radius
  $r_{\rm e}=0.15 \times P$
\end{itemize}

\subsection{Galaxies}
We use two kinds of galaxy distributions: poissonian and clustered. 

... The clustered galaxy distribution (Steph).

\section{Allocation Algorithms}

We use two allocation algorithms

\subsection{Simulated Annealing}
...Simulated Annealing (SA)

\subsection{Local Galaxy Density}
...Local Galaxy Density (LGD)

\section{Results}

The following table summarizes our results

\begin{table}
\begin{tabular}{cccccc}\hline
Fiber & Galaxy & Allocation & \% Total & \% ELG & \% LRG\\
Setup & Distribution & Algorithm & Completeness & Completeness& Completeness\\\hline
DESpec & Poisson & SA & & -& -\\
BB & Poisson & SA & & -& -\\
DESpec & Mock & SA & & &\\
BB & Mock & SA & & &\\
DESpec & Poisson & LGD & 94 & -& -\\
BB & Poisson & LGD & 88& -&-\\
DESpec & Mock & LGD & 90 & 94 & 86\\
BB & Mock & LGD & & &\\\hline
\end{tabular}
\label{Summary of the results.}
\end{table}

\end{document}
